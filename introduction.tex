\section{Einleitung}
Um einen reibungslosen Ablauf der Veranstaltung zu gewährleisten, möchten wir Sie bitten, frühzeitig mit unserem technischen Ansprechpartner alle relevanten Details abzustimmen.

\begin{enumerate}
    \item \textbf{Mikrofonierung und Equipment}  
    Im Tech Rider finden Sie eine detaillierte Auflistung unseres mitgebrachten Equipments, einschließlich Mikrofone, DI-Boxen und unseres IEM-Racks.  
    Bezüglich der Mikrofonierung sind wir flexibel, solange alle für unser IEM-Mix erforderlichen Kanäle bedient werden (Drumset Overheads sind hierbei optional). Der Aufwand den wir betreiben müssen für unser Monitoring ist bei nutzung unserer Mikrofone minimal, bei Fremdmikrofonierung benötigen wir einen Soundcheck um die Mikrofone einzupegeln.

    \item \textbf{Monitoring und In-Ear-Systeme}  
    Wir bringen ein IEM-Rack mit passivem Split sowie Funk- und kabelgebundene Monitoring-Systeme mit.

    \item \textbf{Stromversorgung}  
    Prüfen Sie die Anforderungen an die Stromversorgung, insbesondere die Bereitstellung ausreichender Steckdosen in Bühnennähe, um das gesamte Equipment sicher betreiben zu können.

    \item \textbf{Kommunikation}  
    Der technische Ansprechpartner vor Ort wird gebeten, spätestens einen Tag vor der Veranstaltung mit unserem Technikverantwortlichen, Georg Kranz (Kontakt siehe Titelseite), die finale Abstimmung vorzunehmen.  

\end{enumerate}